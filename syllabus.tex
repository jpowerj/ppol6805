% Options for packages loaded elsewhere
\PassOptionsToPackage{unicode}{hyperref}
\PassOptionsToPackage{hyphens}{url}
\PassOptionsToPackage{dvipsnames,svgnames,x11names}{xcolor}
%
\documentclass[
]{scrartcl}

\usepackage{amsmath,amssymb}
\usepackage{iftex}
\ifPDFTeX
  \usepackage[T1]{fontenc}
  \usepackage[utf8]{inputenc}
  \usepackage{textcomp} % provide euro and other symbols
\else % if luatex or xetex
  \usepackage{unicode-math}
  \defaultfontfeatures{Scale=MatchLowercase}
  \defaultfontfeatures[\rmfamily]{Ligatures=TeX,Scale=1}
\fi
\usepackage{lmodern}
\ifPDFTeX\else  
    % xetex/luatex font selection
\fi
% Use upquote if available, for straight quotes in verbatim environments
\IfFileExists{upquote.sty}{\usepackage{upquote}}{}
\IfFileExists{microtype.sty}{% use microtype if available
  \usepackage[]{microtype}
  \UseMicrotypeSet[protrusion]{basicmath} % disable protrusion for tt fonts
}{}
\makeatletter
\@ifundefined{KOMAClassName}{% if non-KOMA class
  \IfFileExists{parskip.sty}{%
    \usepackage{parskip}
  }{% else
    \setlength{\parindent}{0pt}
    \setlength{\parskip}{6pt plus 2pt minus 1pt}}
}{% if KOMA class
  \KOMAoptions{parskip=half}}
\makeatother
\usepackage{xcolor}
\setlength{\emergencystretch}{3em} % prevent overfull lines
\setcounter{secnumdepth}{-\maxdimen} % remove section numbering
% Make \paragraph and \subparagraph free-standing
\makeatletter
\ifx\paragraph\undefined\else
  \let\oldparagraph\paragraph
  \renewcommand{\paragraph}{
    \@ifstar
      \xxxParagraphStar
      \xxxParagraphNoStar
  }
  \newcommand{\xxxParagraphStar}[1]{\oldparagraph*{#1}\mbox{}}
  \newcommand{\xxxParagraphNoStar}[1]{\oldparagraph{#1}\mbox{}}
\fi
\ifx\subparagraph\undefined\else
  \let\oldsubparagraph\subparagraph
  \renewcommand{\subparagraph}{
    \@ifstar
      \xxxSubParagraphStar
      \xxxSubParagraphNoStar
  }
  \newcommand{\xxxSubParagraphStar}[1]{\oldsubparagraph*{#1}\mbox{}}
  \newcommand{\xxxSubParagraphNoStar}[1]{\oldsubparagraph{#1}\mbox{}}
\fi
\makeatother


\providecommand{\tightlist}{%
  \setlength{\itemsep}{0pt}\setlength{\parskip}{0pt}}\usepackage{longtable,booktabs,array}
\usepackage{calc} % for calculating minipage widths
% Correct order of tables after \paragraph or \subparagraph
\usepackage{etoolbox}
\makeatletter
\patchcmd\longtable{\par}{\if@noskipsec\mbox{}\fi\par}{}{}
\makeatother
% Allow footnotes in longtable head/foot
\IfFileExists{footnotehyper.sty}{\usepackage{footnotehyper}}{\usepackage{footnote}}
\makesavenoteenv{longtable}
\usepackage{graphicx}
\makeatletter
\newsavebox\pandoc@box
\newcommand*\pandocbounded[1]{% scales image to fit in text height/width
  \sbox\pandoc@box{#1}%
  \Gscale@div\@tempa{\textheight}{\dimexpr\ht\pandoc@box+\dp\pandoc@box\relax}%
  \Gscale@div\@tempb{\linewidth}{\wd\pandoc@box}%
  \ifdim\@tempb\p@<\@tempa\p@\let\@tempa\@tempb\fi% select the smaller of both
  \ifdim\@tempa\p@<\p@\scalebox{\@tempa}{\usebox\pandoc@box}%
  \else\usebox{\pandoc@box}%
  \fi%
}
% Set default figure placement to htbp
\def\fps@figure{htbp}
\makeatother

\makeatletter
\@ifpackageloaded{tcolorbox}{}{\usepackage[skins,breakable]{tcolorbox}}
\@ifpackageloaded{fontawesome5}{}{\usepackage{fontawesome5}}
\definecolor{quarto-callout-color}{HTML}{909090}
\definecolor{quarto-callout-note-color}{HTML}{0758E5}
\definecolor{quarto-callout-important-color}{HTML}{CC1914}
\definecolor{quarto-callout-warning-color}{HTML}{EB9113}
\definecolor{quarto-callout-tip-color}{HTML}{00A047}
\definecolor{quarto-callout-caution-color}{HTML}{FC5300}
\definecolor{quarto-callout-color-frame}{HTML}{acacac}
\definecolor{quarto-callout-note-color-frame}{HTML}{4582ec}
\definecolor{quarto-callout-important-color-frame}{HTML}{d9534f}
\definecolor{quarto-callout-warning-color-frame}{HTML}{f0ad4e}
\definecolor{quarto-callout-tip-color-frame}{HTML}{02b875}
\definecolor{quarto-callout-caution-color-frame}{HTML}{fd7e14}
\makeatother
\makeatletter
\@ifpackageloaded{caption}{}{\usepackage{caption}}
\AtBeginDocument{%
\ifdefined\contentsname
  \renewcommand*\contentsname{Table of contents}
\else
  \newcommand\contentsname{Table of contents}
\fi
\ifdefined\listfigurename
  \renewcommand*\listfigurename{List of Figures}
\else
  \newcommand\listfigurename{List of Figures}
\fi
\ifdefined\listtablename
  \renewcommand*\listtablename{List of Tables}
\else
  \newcommand\listtablename{List of Tables}
\fi
\ifdefined\figurename
  \renewcommand*\figurename{Figure}
\else
  \newcommand\figurename{Figure}
\fi
\ifdefined\tablename
  \renewcommand*\tablename{Table}
\else
  \newcommand\tablename{Table}
\fi
}
\@ifpackageloaded{float}{}{\usepackage{float}}
\floatstyle{ruled}
\@ifundefined{c@chapter}{\newfloat{codelisting}{h}{lop}}{\newfloat{codelisting}{h}{lop}[chapter]}
\floatname{codelisting}{Listing}
\newcommand*\listoflistings{\listof{codelisting}{List of Listings}}
\makeatother
\makeatletter
\makeatother
\makeatletter
\@ifpackageloaded{caption}{}{\usepackage{caption}}
\@ifpackageloaded{subcaption}{}{\usepackage{subcaption}}
\makeatother

\usepackage{bookmark}

\IfFileExists{xurl.sty}{\usepackage{xurl}}{} % add URL line breaks if available
\urlstyle{same} % disable monospaced font for URLs
\hypersetup{
  pdftitle={PPOL 6805 / DSAN 6750 Syllabus - Fall 2024},
  pdfauthor={Prof.~Jeff Jacobs},
  colorlinks=true,
  linkcolor={blue},
  filecolor={Maroon},
  citecolor={Blue},
  urlcolor={Blue},
  pdfcreator={LaTeX via pandoc}}


\title{PPOL 6805 / DSAN 6750}
\usepackage{etoolbox}
\makeatletter
\providecommand{\subtitle}[1]{% add subtitle to \maketitle
  \apptocmd{\@title}{\par {\large #1 \par}}{}{}
}
\makeatother
\subtitle{Geographic Information Systems (GIS) for Spatial Data
Science \\ \textit{Wednesdays 6:30-9pm, Car Barn 204}}
\author{Prof.~Jeff
Jacobs \\[-0.2em] \normalsize{\href{mailto:jj1088@georgetown.edu}{\texttt{jj1088@georgetown.edu}}}}
\date{\normalsize{Fall 2024, Georgetown University}}
% 
\begin{document}
\maketitle


Welcome to the Fall 2024 version of \textbf{Geographic Information
Systems (GIS) for Spatial Data Science} at Georgetown University! Please
note that the most up-to-date version of this syllabus will always be
available at \href{https://jjacobs.me/ppol6805}{jjacobs.me/ppol6805}

\begin{tcolorbox}[enhanced jigsaw, left=2mm, colbacktitle=quarto-callout-note-color!10!white, opacitybacktitle=0.6, title=\textcolor{quarto-callout-note-color}{\faInfo}\hspace{0.5em}{Course Numbering}, toprule=.15mm, breakable, opacityback=0, colback=white, colframe=quarto-callout-note-color-frame, arc=.35mm, leftrule=.75mm, bottomtitle=1mm, toptitle=1mm, rightrule=.15mm, bottomrule=.15mm, coltitle=black, titlerule=0mm]

\begin{itemize}
\tightlist
\item
  For \emph{Public Policy} students (McCourt): the course number is
  \textbf{PPOL 6805}
\item
  For \emph{Data Science and Analytics} students: the course number is
  \textbf{DSAN 6750}
\end{itemize}

\end{tcolorbox}

\subsection{Course Staff and Office
Hours}\label{course-staff-and-office-hours}

\begin{itemize}
\tightlist
\item
  \textbf{Prof.~Jeff Jacobs},
  \href{mailto:jj1088@georgetown.edu}{\texttt{jj1088@georgetown.edu}}:
  \emph{Schedule office hour slots at
  \href{https://jjacobs.me/meet}{\texttt{jjacobs.me/meet}}}

  \begin{itemize}
  \tightlist
  \item
    Mondays 4:30-6:30pm
  \item
    Tuesdays 3-5:30pm
  \item
    \emph{(Please try to schedule at least 8 hours in advance, and let
    me know briefly what you'd like to discuss, so I have time to
    prepare)}
  \end{itemize}
\item
  \textbf{TA Billy McGloin},
  \href{mailto:wtm30@georgetown.edu}{\texttt{wtm30@georgetown.edu}}

  \begin{itemize}
  \tightlist
  \item
    By appointment (Email)
  \end{itemize}
\end{itemize}

\subsection{Course Description}\label{course-description}

This course provides students with an overview of Geographic Information
Systems (GIS), encompassing both general principles of geospatial data
analysis and their applications for studying important issues of climate
change, international conflict, economic development, and urban
planning, among other areas of application.

The beginning of the course emphasizes fundamentals of GIS design and
use, such as projection systems, raster and vector data, and efficient
representation and storage of geospatial data. As students become more
comfortable with these foundations, the course will shift to a greater
emphasis on applications during the second half of the semester.
Particular emphasis will be placed on effective visualization of spatial
data, through creation of static publication-quality maps as well as
interactive maps for web applications and data dashboards.

The course will utilize libraries from R, Python, and JavaScript as
needed, so experience using any of these languages will be helpful, but
is not required. Pre-requisites \emph{(for PPOL students)}: PPOL 564 or
PPOL 670.

\subsection{Assignment Structure}\label{assignment-structure}

On the basis of the guidelines we've developed for courses offered
through DSAN, this course will have one in-class midterm but \textbf{no}
final exam! Instead, you will work on a final project throughout the
second half of the course. Each of the four units will involve a
homework assignment, and final grades will be determined using the
following weighting scheme:

\begin{longtable}[]{@{}ll@{}}
\toprule\noalign{}
Category & Percent of Final Grade \\
\midrule\noalign{}
\endhead
\bottomrule\noalign{}
\endlastfoot
In-Class Midterm (Nov 6) & 30\% \\
Final Project & 30\% \\
Homeworks & 40\% \\
\textbf{HW1}: GIS Concepts & 10\% \\
\textbf{HW2}: Unary Operations & 10\% \\
\textbf{HW3}: Binary Operations and Spatial Joins & 10\% \\
\textbf{HW4}: Autocorrelation, Clustering, and Point Processes & 10\% \\
\textbf{HW5A}: Point Hypothesis Evaluation & {[}+50 Bonus Points{]} \\
\textbf{HW5B}: Spatial Regression & {[}+50 Bonus Points{]} \\
\end{longtable}

The course does not have any ``official'' prerequisites, but a general
comfort with \textbf{R} and/or \textbf{Python} is strongly recommended.
If you have never used Python before, however (or if you haven't used it
in a while and feel like your skills are rusty), you can browse the
materials on the \href{resources.qmd}{Resources page}!

\subsection{Course Topics / Calendar}\label{course-topics-calendar}

The following is a rough map of what we will work through together
throughout the semester; given that \textbf{everyone learns at a
different pace}, my aim is to leave us with a good amount of
\textbf{flexibility} in terms of how much time we spend on each topic.

If I find that it takes me longer than a week to convey a certain topic
in sufficient depth, for example, then I view it as a strength rather
than a weakness of the course that we can then rearrange the calendar
below by adding an extra week on that particular topic! Similarly, if it
seems like I am spending too much time on a topic, to the point that
students seem bored or impatient to move onto the next topic, we can
move a topic intended for the next week to the current week!

If you find any discrepancies between this schedule and Georgetown's
\href{https://registrar.georgetown.edu/academic-calendar/maincampus/}{official
calendar}, please let me know.

\begin{longtable}[]{@{}
  >{\raggedright\arraybackslash}p{(\linewidth - 6\tabcolsep) * \real{0.2500}}
  >{\raggedright\arraybackslash}p{(\linewidth - 6\tabcolsep) * \real{0.1000}}
  >{\raggedright\arraybackslash}p{(\linewidth - 6\tabcolsep) * \real{0.2000}}
  >{\raggedright\arraybackslash}p{(\linewidth - 6\tabcolsep) * \real{0.4500}}@{}}
\toprule\noalign{}
\begin{minipage}[b]{\linewidth}\raggedright
Unit
\end{minipage} & \begin{minipage}[b]{\linewidth}\raggedright
Week
\end{minipage} & \begin{minipage}[b]{\linewidth}\raggedright
Date
\end{minipage} & \begin{minipage}[b]{\linewidth}\raggedright
Topic
\end{minipage} \\
\midrule\noalign{}
\endhead
\bottomrule\noalign{}
\endlastfoot
\textbf{Unit 1}: GIS Concepts & 1 & Aug 28 & \href{./w01/}{Introduction
to GIS} \\
& 2 & Sep 4 & \begin{minipage}[t]{\linewidth}\raggedright
\href{./w02/}{How Do Maps Work?}\\
\emph{HW1 Release}\strut
\end{minipage} \\
\textbf{Unit 2:} Geospatial Operations & 3 & Sep 11 &
\begin{minipage}[t]{\linewidth}\raggedright
\href{./w03/}{Unary Operations}\\
\emph{{[}Deliverable{]} HW1: GIS Concepts}\strut
\end{minipage} \\
& & \emph{Sep 13 (Fri)} & \emph{HW2 Release} \\
& 4 & Sep 18 & \href{./w04/}{Binary Operations} \\
& & \emph{Sep 20 (Fri), 5:59pm} & \emph{{[}Deliverable{]} HW2: Unary
Operations} \\
\textbf{Unit 3:} Spatial Data Science I: Foundations & 5 & Sep 25 &
\begin{minipage}[t]{\linewidth}\raggedright
\href{./w05/}{Spatial Data Science!}\\
\emph{HW3 Release}\strut
\end{minipage} \\
& 6 & Oct 2 & \begin{minipage}[t]{\linewidth}\raggedright
\href{./w06/}{Random Fields and Spatial Autocorrelation}\\
\emph{HW4 Release}\strut
\end{minipage} \\
& & \emph{Oct 4 (Fri), 5:59pm} & \emph{{[}Deliverable{]} HW3: Binary
Operations and Spatial Joins} \\
\textbf{Unit 4:} Spatial Data Science II: Methods & 7 & Oct 9 &
\href{./w07/}{Point Processes, Clustering, and Regularity} \\
& 8 & Oct 16 & \href{./w08/}{Null Models and Marked Point Processes} \\
& 9 & Oct 23 & \href{./w09/}{Evaluating Spatial Hypotheses I: Point
Data} \\
& 10 & Oct 30 & \begin{minipage}[t]{\linewidth}\raggedright
\href{./w10/}{Evaluating Spatial Hypotheses II: Areal Data}\\
\emph{{[}Deliverable, 11:59pm{]} HW4: Autocorrelation, Clustering, and
Point Processes}\strut
\end{minipage} \\
& & \emph{Nov 1 (Fri)} & \emph{Midterm Practice Problems Release} \\
& 11 & Nov 6 & \href{./w11/}{In-Class Midterm} \\
\textbf{Unit 5}: Final Projects & 12 & Nov 13 & \href{./w12/}{Tools for
Final Projects} \\
& 13 & Nov 20 & \href{./w13/}{In-Class Office Hours} \\
& & \emph{Nov 27} & \emph{No Class (Fall Break)} \\
& 14 & Dec 4 & \href{./w14/}{Final Presentations I} \\
& 15 & Dec 11 & \href{./w15/}{Final Presentations II} \\
& & \emph{Dec 13 (Friday), 5:59pm} & \emph{{[}Deliverable{]} Final
Project} \\
\end{longtable}

\subsection{Assignment Distribution, Submission, and
Grading}\label{assignment-distribution-submission-and-grading}

The programming assignments for the course will be managed through
\href{https://posit.cloud/spaces/547750/join?access_code=pxoi5atXz19sdp1N9e2X2qUbtxGV8yqRTtN5MeeD}{Posit.Cloud}
This means that, to work on and submit the assignments, you will use the
following workflow:

\begin{enumerate}
\def\labelenumi{\arabic{enumi}.}
\tightlist
\item
  Click the ``Start'' button for the assignment within the Posit.Cloud
  interface, which should immediately create and display a new
  in-browser RStudio workspace for you.
\item
  Work on the problems within the project's \texttt{.qmd} file, saving
  your progress early and often! You can \textbf{try things out} or
  \textbf{create drafts} of your solutions locally if you'd like (for
  example, in VSCode or JupyterLab or any other IDE), but the
  \texttt{.qmd} file within your RStudio project will be considered your
  ``official'' submission for each assignment.
\end{enumerate}

\subsection{Late Policy}\label{late-policy}

After the due date, for each \textbf{homework} assignment, you will have
a grace period of 24 hours to submit the assignment without a lateness
penalty. After this 24 hour grace period, late penalties will be applied
up until 66 hours after the due date. Specifically, late penalties will
be applied based on the following scale (unless you obtain an excused
lateness from one of the instructional staff!):

\begin{itemize}
\tightlist
\item
  \textbf{0 to 24 hours} after due date: no penalty
\item
  \textbf{24 to 30 hours} after due date: 2.5\% penalty
\item
  \textbf{30 to 42 hours} after due date: 5\% penalty
\item
  \textbf{42 to 54 hours} after due date: 10\% penalty
\item
  \textbf{54 to 66 hours} after due date: 20\% penalty
\item
  \textbf{More than 66 hours} after due date: Assignment submissions no
  longer accepted (without instructor approval)
\end{itemize}

\subsection{Final Letter Grade
Determination}\label{final-letter-grade-determination}

Once all assignments have been graded, we will compute your final
\textbf{numeric grade} according to the above weighting, rounded to two
decimal places. The \textbf{letter grade} that we report to Georgetown
on the basis of this numeric grade will then follow the DSAN letter
grade policy as follows, where the start and end points for each range
are \textbf{inclusive}:

\begin{longtable}[]{@{}rrl@{}}
\toprule\noalign{}
Range Start & Range End & Letter Grade \\
\midrule\noalign{}
\endhead
\bottomrule\noalign{}
\endlastfoot
92.50 & 100.00 & A \\
89.50 & 92.49 & A- \\
87.99 & 89.49 & B+ \\
81.50 & 87.98 & B \\
79.50 & 81.49 & B- \\
69.50 & 79.49 & C \\
59.50 & 69.49 & D \\
0.00 & 59.49 & F \\
\end{longtable}

\subsection{Title IX/Sexual Misconduct
Statement}\label{title-ixsexual-misconduct-statement}

Georgetown University and its faculty are committed to supporting
survivors and those impacted by sexual misconduct, which includes sexual
assault, sexual harassment, relationship violence, and stalking.
Georgetown requires faculty members, unless otherwise designated as
confidential, to report all disclosures of sexual misconduct to the
University Title IX Coordinator or a Deputy Title IX Coordinator.

If you disclose an incident of sexual misconduct to a professor in or
outside of the classroom (with the exception of disclosures in papers),
that faculty member must report the incident to the Title IX
Coordinator, or Deputy Title IX Coordinator. The coordinator will, in
turn, reach out to the student to provide support, resources, and the
option to meet. {[}Please note that the student is not required to meet
with the Title IX coordinator.{]}. More information about reporting
options and resources can be found in the Sexual Misconduct Resource
Center.

If you would prefer to speak to someone confidentially, Georgetown has a
number of fully confidential professional resources that can provide
support and assistance. These resources include:

\begin{itemize}
\tightlist
\item
  \emph{Health Education Services for Sexual Assault Response and
  Prevention}: Confidential email \texttt{sarp@georgetown.edu}
\item
  \emph{Counseling and Psychiatric Services (CAPS)}: 202-687-6985

  \begin{itemize}
  \tightlist
  \item
    After hours you can call 833-960-3006 to reach Fonemed, a telehealth
    service, and ask for the on-call CAPS clinician
  \end{itemize}
\end{itemize}

\subsection{GSAS and McCourt Resources and
Policies}\label{gsas-and-mccourt-resources-and-policies}

You can find a collection of relevant resources and policies for
students on the GSAS website, and the Provost's policy on accommodating
students' religious observances on the Campus Ministry website.

You can also make use of the Student Academic Resource Center. In
particular, within the Resource Center there is a link to Georgetown's
Disability Support page. If you believe you have a disability, you can
contact the Academic Resource Center (\texttt{arc@georgetown.edu}) for
further information. The ARC is located in the Leavey Center, Suite 335
(202-687-8354), and it is the campus office responsible for reviewing
documentation provided by students with disabilities and for determining
reasonable accommodations in accordance with the Americans with
Disabilities Act (ADA) and University policies.

\subsubsection{McCourt Academic Integrity
Policy}\label{mccourt-academic-integrity-policy}

McCourt School students are expected to uphold the academic policies set
forth by Georgetown University and the Graduate School of Arts and
Sciences. Students should therefore familiarize themselves with all the
rules, regulations, and procedures relevant to their pursuit of a
Graduate School degree. The relevant policies are listed at
\href{https://sites.google.com/a/georgetown.edu/gsas-graduate-bulletin/vi-academic-integrity-policies-procedures}{this
link}.




\end{document}
